\documentclass{article}
\usepackage{graphicx}
\usepackage [spanish] {babel} 
\usepackage [T1]{fontenc}
\usepackage [latin1]{inputenc}

\begin{document}

\title{Tarea 5 Metodos Computacionales}
\author{Maria Alejandra Gonzalez - 201215953}

\maketitle

\begin{abstract}
En este archivo PDF se muestran todos los resultados obtenidos en la realizacion de la tarea final del curso.
Inicialmente se encuentran las graficas obtenidas en el estudio de los canales ionicos de unas particulas aleatoriamente distribuidas, y finalmente las graficas de resultados del analisis del circuito RC.
\end{abstract}

\section{Canales Ionicos}
Distribucion de las moleculas que componen el canal ionico, de color rojo se vera el circulo maximo encontrado para cada uno de los casos.

\subsection{Canales ionicos 1}
Para este caso la grafica asociada es la siguiente:
\begin{figure}[h!]
   \centering
    \includegraphics[width=3.0in]{Radio_ionico.jpg}
    \caption{Canales ionicos 1}
    \label{1}
\end{figure}

Luego el histograma asociado es el siguiente:

\begin{figure}[h!]
   \centering
    \includegraphics[width=3.0in]{hist1.jpg}
    \caption{Canales ionicos 1-histograma}
    \label{2}
\end{figure}

\subsection{Canales ionicos 2}
Para este caso la grafica asociada es la siguiente:
\begin{figure}[h!]
   \centering
    \includegraphics[width=3.0in]{Radio_ion1.jpg}
    \caption{Canales ionicos 2}
    \label{3}
\end{figure}
Luego el histograma asociado es el siguiente:
\begin{figure}[h!]
   \centering
    \includegraphics[width=3.0in]{hist2.jpg}
    \caption{Canales ionicos 2-histograma}
    \label{4}
\end{figure}

\section{Carga de un circuito RC}
Para este ejercicio se encontraron los valores maximos de C y R para unos datos experimentales de un circuito RC donde teniamos Q y el tiempo.
\begin{figure}[h!]
   \centering
    \includegraphics[width=3.0in]{experimentales.jpg}
    \caption{datos experimentales}
    \label{5}
\end{figure}
Luego el histograma asociado es
\begin{figure}[h!]
   \centering
    \includegraphics[width=3.0in]{histcircuitos.jpg}
    \caption{datos experimentales}
    \label{6}
\end{figure}

\end{document}